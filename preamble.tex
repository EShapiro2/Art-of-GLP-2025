% preamble.tex - Art of GLP Book Preamble

% Essential packages
\usepackage{graphicx}
\usepackage{amsmath,amssymb,amsfonts}
\usepackage{amsthm}
\usepackage{xspace}
\usepackage{hyperref}
\usepackage{cleveref}
\usepackage{xcolor}
\usepackage{enumitem}
\usepackage{thmtools}
\usepackage{mdframed}
\usepackage{marginnote}

% Additional packages from paper
\usepackage{multirow}
\usepackage{mathtools}
\usepackage{relsize}
\usepackage{bm}
\usepackage{verbatimbox}
\usepackage{wrapfig}

% Algorithm packages
\usepackage{algorithm}
\usepackage[noend]{algpseudocode}

% Code listings
\usepackage{listings}
\lstset{
  basicstyle=\ttfamily\small,
  frame=single,
  breaklines=true,
  columns=fullflexible
}

% Memoir-specific settings
\chapterstyle{veelo}
\setlength{\beforechapskip}{50pt}
\setlength{\afterchapskip}{40pt}

% Page layout
\raggedbottom

% Bibliography style
\bibliographystyle{plain}

%% ============================================
%% Text Commands and Abbreviations
%% ============================================

\newcommand{\GLP}{\textsc{GLP}\xspace}
\newcommand{\lp}{logic programs\xspace}
\newcommand{\cp}{Concurrent Prolog\xspace}
\newcommand{\scl}{\textsc{scl}\xspace}
\newcommand{\gsn}{\textsc{gsn}\xspace}

% Paragraph and emphasis commands
\newcommand{\mypara}[1]{\paragraph{#1}}
\newcommand{\temph}[1]{\textbf{#1}}
\newcommand{\remove}[1]{}

% Author comment commands (for drafts)
\newcommand{\udi}[1]{\textcolor{blue}{[Udi says: #1]}}
\newcommand{\claude}[1]{\textcolor{red}{[Claude: #1]}}

%% ============================================
%% Math Commands
%% ============================================

\newcommand{\calV}{\mathcal{V}}
\newcommand{\calG}{\mathcal{G}}
\newcommand{\calT}{\mathcal{T}}
\newcommand{\calF}{\mathcal{F}}
\newcommand{\calR}{\mathbb{R}}
\newcommand{\calN}{\mathbb{N}}
\newcommand{\calA}{\mathcal{A}}
\newcommand{\V}{\mathcal{V}}
\newcommand{\calS}{\mathcal{S}}
\newcommand{\calL}{\mathcal{L}}
\newcommand{\calP}{\mathcal{P}}
\newcommand{\calM}{\mathcal{M}}
\newcommand{\calE}{\mathcal{E}}
\newcommand{\calB}{\mathcal{B}}
\newcommand{\calC}{\mathcal{C}}
\newcommand{\calX}{\mathcal{X}}
\newcommand{\calD}{\mathcal{D}}

%% ============================================
%% Roman Numeral Abbreviations
%% ============================================

\newcommand{\ia}{\textit{i}}
\newcommand{\ib}{\textit{ii}}
\newcommand{\ic}{\textit{iii}}
\newcommand{\id}{\textit{iv}}
\newcommand{\iie}{\textit{v}}
\newcommand{\iif}{\textit{vi}}
\newcommand{\iiv}{\textit{iv}}
\newcommand{\iv}{\textit{v}}

%% ============================================
%% Program Counter
%% ============================================

\newcounter{programcounter}[chapter]
\renewcommand{\theprogramcounter}{\thechapter.\arabic{programcounter}}
\newcommand{\Program}[1]{%
  \refstepcounter{programcounter}%
  \medskip\noindent\textbf{Program \theprogramcounter: #1}\par\nopagebreak
}

%% ============================================
%% List Settings
%% ============================================

\setlist{nosep, leftmargin=*}
\setlist{itemsep=1pt, topsep=3pt, leftmargin=*}

%% ============================================
%% Theorem Environments (for main text)
%% ============================================

\newtheorem{observation}{Observation}[chapter]
\newtheorem{definition}{Definition}[chapter]
\newtheorem{proposition}[definition]{Proposition}
\newtheorem{theorem}[definition]{Theorem}
\newtheorem{lemma}[definition]{Lemma}
\newtheorem{corollary}[definition]{Corollary}
\newtheorem{example}{Example}[chapter]
\newtheorem{remark}{Remark}[chapter]

%% ============================================
%% Restatable (from thmtools)
%% ============================================

\declaretheorem[name=Proposition,numberwithin=chapter]{restatable}

%% ============================================
%% Include Formal Track Definitions
%% ============================================

% formal.tex - Formal Track Infrastructure
%
% This file defines the infrastructure for the "formal track" of the book.
% Formal content appears in shaded boxes that readers can optionally skip.

%% ============================================
%% Counter for Formal Boxes
%% ============================================

\newcounter{formalbox}[chapter]
\renewcommand{\theformalbox}{\thechapter.\arabic{formalbox}}

%% ============================================
%% Formal Box Environment
%% ============================================

% The formal environment creates a shaded box for rigorous content.
% Readers who prefer intuition can skip these boxes entirely.

\newenvironment{formal}[1]{%
  \refstepcounter{formalbox}%
  \begin{mdframed}[
    backgroundcolor=gray!10,
    linewidth=1pt,
    linecolor=gray!50,
    innertopmargin=10pt,
    innerbottommargin=10pt,
    innerleftmargin=10pt,
    innerrightmargin=10pt,
    skipabove=\baselineskip,
    skipbelow=\baselineskip
  ]
  \noindent\textbf{Formal \theformalbox: #1}\par\smallskip
}{%
  \end{mdframed}
}

%% ============================================
%% Theorem Environments for Formal Track
%% ============================================

% These environments are used inside formal boxes for definitions,
% propositions, theorems, and lemmas with formal proofs.

\newtheorem{fdef}{Definition}[chapter]
\newtheorem{fprop}[fdef]{Proposition}
\newtheorem{fthm}[fdef]{Theorem}
\newtheorem{flem}[fdef]{Lemma}
\newtheorem{fcor}[fdef]{Corollary}

%% ============================================
%% Margin Indicator for Formal Content
%% ============================================

% Use \formalmargin at the start of formal sections to place
% a "FORMAL" indicator in the margin.

\newcommand{\formalmargin}{\marginnote{\textsc{formal}}}

%% ============================================
%% Proof Environment Customization
%% ============================================

% Ensure proofs work well inside formal boxes
\renewcommand{\qedsymbol}{$\blacksquare$}

