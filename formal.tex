% formal.tex - Formal Track Infrastructure
%
% This file defines the infrastructure for the "formal track" of the book.
% Formal content appears in shaded boxes that readers can optionally skip.

%% ============================================
%% Counter for Formal Boxes
%% ============================================

\newcounter{formalbox}[chapter]
\renewcommand{\theformalbox}{\thechapter.\arabic{formalbox}}

%% ============================================
%% Formal Box Environment
%% ============================================

% The formal environment creates a shaded box for rigorous content.
% Readers who prefer intuition can skip these boxes entirely.

\newenvironment{formal}[1]{%
  \refstepcounter{formalbox}%
  \begin{mdframed}[
    backgroundcolor=gray!10,
    linewidth=1pt,
    linecolor=gray!50,
    innertopmargin=10pt,
    innerbottommargin=10pt,
    innerleftmargin=10pt,
    innerrightmargin=10pt,
    skipabove=\baselineskip,
    skipbelow=\baselineskip
  ]
  \noindent\textbf{Formal \theformalbox: #1}\par\smallskip
}{%
  \end{mdframed}
}

%% ============================================
%% Theorem Environments for Formal Track
%% ============================================

% These environments are used inside formal boxes for definitions,
% propositions, theorems, and lemmas with formal proofs.

\newtheorem{fdef}{Definition}[chapter]
\newtheorem{fprop}[fdef]{Proposition}
\newtheorem{fthm}[fdef]{Theorem}
\newtheorem{flem}[fdef]{Lemma}
\newtheorem{fcor}[fdef]{Corollary}

%% ============================================
%% Margin Indicator for Formal Content
%% ============================================

% Use \formalmargin at the start of formal sections to place
% a "FORMAL" indicator in the margin.

\newcommand{\formalmargin}{\marginnote{\textsc{formal}}}

%% ============================================
%% Proof Environment Customization
%% ============================================

% Ensure proofs work well inside formal boxes
\renewcommand{\qedsymbol}{$\blacksquare$}
