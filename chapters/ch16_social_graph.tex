% chapters/ch16_social_graph.tex - Social Graph

\chapter{The Grassroots Social Graph}
\label{ch:social-graph}

This chapter presents the grassroots social graph---the foundation for all grassroots platforms.

\section{Graph Structure}

The grassroots social graph is a distributed graph where:
\begin{itemize}
\item Nodes represent cryptographically-identified people
\item Edges represent authenticated friendships
\item Connected components arise spontaneously and merge through befriending
\end{itemize}

\section{Node Representation}

Each person runs a node on their personal device. A node contains:
\begin{itemize}
\item A keypair (public/private) for identification
\item A friend list (authenticated connections)
\item The \GLP runtime
\end{itemize}

\section{Edge Representation}

An edge between two nodes represents:
\begin{itemize}
\item Mutual authentication (each knows the other's public key)
\item A bidirectional communication channel
\item Trust relationship for friend-mediated introductions
\end{itemize}

\section{Graph Properties}

[To be developed: formal properties of the social graph]

\section{Exercises}

[To be developed]
