\documentclass[11pt,a4paper,openright]{memoir}

%% ============================================
%% PACKAGES
%% ============================================

\usepackage{graphicx}
\usepackage{amsmath,amssymb,amsfonts}
\usepackage{amsthm}
\usepackage{xspace}
\usepackage{hyperref}
\usepackage{cleveref}
\usepackage{xcolor}
\usepackage{enumitem}
\usepackage{mdframed}
\usepackage{multirow}
\usepackage{mathtools}
\usepackage{algorithm}
\usepackage[noend]{algpseudocode}

%% ============================================
%% MEMOIR SETTINGS
%% ============================================

\chapterstyle{madsen}
\setlength{\beforechapskip}{50pt}
\setlength{\afterchapskip}{40pt}
\raggedbottom

%% ============================================
%% COMMANDS (from paper)
%% ============================================

\newcommand{\GLP}{\textsc{GLP}\xspace}
\newcommand{\mypara}[1]{\smallskip\noindent\textbf{#1.}}
\newcommand{\temph}[1]{\textbf{#1}}
\newcommand{\remove}[1]{}

% Math commands
\newcommand{\calV}{\mathcal{V}}
\newcommand{\calG}{\mathcal{G}}
\newcommand{\calT}{\mathcal{T}}
\newcommand{\calF}{\mathcal{F}}
\newcommand{\calR}{\mathbb{R}}
\newcommand{\calN}{\mathbb{N}}
\newcommand{\calA}{\mathcal{A}}
\newcommand{\V}{\mathcal{V}}
\newcommand{\calS}{\mathcal{S}}
\newcommand{\calL}{\mathcal{L}}
\newcommand{\calP}{\mathcal{P}}
\newcommand{\calM}{\mathcal{M}}
\newcommand{\calE}{\mathcal{E}}
\newcommand{\calB}{\mathcal{B}}
\newcommand{\calC}{\mathcal{C}}
\newcommand{\calX}{\mathcal{X}}
\newcommand{\calD}{\mathcal{D}}

% Program counter
\newcounter{programcounter}[chapter]
\renewcommand{\theprogramcounter}{\thechapter.\arabic{programcounter}}
\newcommand{\Program}[1]{%
  \refstepcounter{programcounter}%
  \medskip\noindent\textbf{Program \theprogramcounter: #1}\par\nopagebreak
}

% List settings
\setlist{nosep, leftmargin=*}

% Theorem environments
\newtheorem{observation}{Observation}[chapter]
\newtheorem{definition}{Definition}[chapter]
\newtheorem{proposition}[definition]{Proposition}
\newtheorem{theorem}[definition]{Theorem}
\newtheorem{lemma}[definition]{Lemma}

%% ============================================
%% FORMAL TRACK ENVIRONMENT
%% ============================================

\newcounter{formalbox}[chapter]
\renewcommand{\theformalbox}{\thechapter.\arabic{formalbox}}

\newenvironment{formal}[1]{%
  \refstepcounter{formalbox}%
  \begin{mdframed}[
    backgroundcolor=gray!10,
    linewidth=1pt,
    linecolor=gray!50,
    innertopmargin=10pt,
    innerbottommargin=10pt,
    innerleftmargin=10pt,
    innerrightmargin=10pt,
    skipabove=\baselineskip,
    skipbelow=\baselineskip
  ]
  \noindent\textbf{Formal \theformalbox: #1}\par\smallskip
}{%
  \end{mdframed}
}

%% ============================================
%% DOCUMENT
%% ============================================

\begin{document}

%% ============================================
%% TITLE PAGE
%% ============================================

\thispagestyle{empty}
\begin{center}
\vspace*{3cm}

{\Huge\bfseries The Art of\\[0.5cm] Grassroots Logic Programming}

\vspace{3cm}

{\Large Ehud Shapiro}

\vspace{1cm}

{\large London School of Economics\\and\\Weizmann Institute of Science}

\vfill

{\large 2025}

\end{center}
\cleardoublepage

%% ============================================
%% FRONT MATTER
%% ============================================

\frontmatter

\chapter{Preface}

This book presents \GLP, a secure, multiagent, concurrent logic programming language designed for implementing grassroots platforms.

[Preface to be written]

\chapter{How to Read This Book}

This book has two parallel tracks: an \emph{informal track} and a \emph{formal track}.

\section*{The Informal Track}

The informal track provides intuitive explanations, examples, and programming techniques. It is comprehensive and self-contained. Most readers should follow this track.

\section*{The Formal Track}

The formal track provides precise mathematical definitions and proofs. It appears in shaded boxes labeled ``Formal X.Y'' throughout the text.

\begin{formal}{Example Formal Box}
This is what a formal box looks like. Readers who prefer intuition can skip these boxes entirely.
\end{formal}

\section*{Book Organization}

The book is organized in two parts:

\textbf{Part I: Concurrent GLP} covers single-agent concurrent logic programming, including the core language, programming techniques, and simulation of multiagent systems.

\textbf{Part II: Multiagent GLP} covers distributed multiagent systems, security, and grassroots protocols. [To be developed]

\tableofcontents

%% ============================================
%% MAIN MATTER
%% ============================================

\mainmatter

%% ============================================
%% PART I: CONCURRENT GLP
%% ============================================

\part{Concurrent GLP}

\chapter{Introduction}

[Introduction chapter to be developed]

\section{The Grassroots Vision}

Grassroots platforms are distributed applications run by cryptographically-identified people on their networked personal devices, where multiple disjoint platform instances emerge independently and coalesce when they interoperate.

\section{Why GLP?}

\GLP extends logic programs with paired single-reader/single-writer variables, providing secure communication channels among cryptographically-identified people.

\section{A First Example: Merge}

\Program{Fair Stream Merger}
\begin{verbatim}
merge([X|Xs],Ys,[X?|Zs?]) :- merge(Ys?,Xs?,Zs).
merge(Xs,[Y|Ys],[Y?|Zs?]) :- merge(Xs?,Ys?,Zs).
merge([],[],[]).
\end{verbatim}

\chapter{Logic Programs}

[Logic programs chapter to be developed]

\chapter{GLP Core}

[GLP core concepts chapter to be developed]

\chapter{GLP Computation}

[GLP computation chapter to be developed]

\chapter{Streams}

[Streams chapter to be developed]

\chapter{Programming Techniques}

[Programming techniques chapter to be developed]

\chapter{Simulating Multiagent Systems}

[Plays and multiagent simulation chapter to be developed]

%% ============================================
%% PART II: MULTIAGENT GLP (placeholder)
%% ============================================

\part{Multiagent GLP}

[Part II to be developed]

%% ============================================
%% BACK MATTER
%% ============================================

\backmatter

\bibliographystyle{plain}
\bibliography{bib}

\end{document}
