% frontmatter/preface.tex - Preface

\chapter{Preface}

This book presents \GLP, a secure, multiagent, concurrent logic programming language designed for implementing grassroots platforms---distributed applications run by cryptographically-identified people on their networked personal devices.

\section*{A Personal Journey}

The journey to \GLP began decades ago with Concurrent Prolog and the Fifth Generation Computer Systems project. The vision then was to harness the power of logic programming for parallel computation. That vision evolved, and the core ideas matured through Flat Concurrent Prolog (FCP) and its theoretical foundations.

\GLP represents the culmination of this journey, addressing a new challenge: how can ordinary people, using only their personal devices, create and operate distributed platforms without relying on centralized servers or trusted third parties?

\section*{Relationship to Earlier Works}

Readers familiar with \emph{The Art of Prolog} will recognize the pedagogical approach: we build understanding progressively, from simple concepts to sophisticated applications. However, where that book focused on sequential logic programming, this book embraces concurrency as fundamental.

Readers of \emph{Concurrent Prolog: Collected Papers} will find the theoretical foundations familiar, but the focus here is practical: how to write programs that actually work in grassroots settings, with all the challenges of security, identity, and distributed coordination.

\section*{The Grassroots Vision}

The grassroots vision is simple but radical: platforms owned and operated by their users, not by corporations. Social networks without surveillance capitalism. Currencies without banks. Democratic institutions without centralized control.

\GLP provides the programming foundation for this vision. Its single-reader/single-writer discipline ensures that distributed computations remain secure. Its cryptographic integration enables identity verification without trusted authorities. Its concurrent semantics naturally express the parallelism inherent in grassroots systems.

\section*{Acknowledgments}

[To be written]

\vspace{1cm}
\noindent
\textit{Rehovot, 2025}
