% appendices/app_solutions.tex - Exercise Solutions

\chapter{Selected Exercise Solutions}
\label{app:solutions}

This appendix provides solutions to selected exercises from the main text.

\section{Chapter~\ref{ch:streams}: Streams}

\subsection{Exercise~\ref{ex:biased-merge}: Biased Merge}

Write \verb|biased_merge(Xs, Ys, Zs)| that takes all available elements from \verb|Xs| before taking from \verb|Ys|.

\begin{verbatim}
% Take from Xs while it has elements ready
biased_merge([X|Xs], Ys, [X?|Zs?]) :- biased_merge(Xs?, Ys?, Zs).

% When Xs tail is unbound (writer), take from Ys
biased_merge(Xs, [Y|Ys], [Y?|Zs?]) :- writer(Xs?) |
    biased_merge(Xs?, Ys?, Zs).

% Termination
biased_merge([], [], []).
biased_merge([], [Y|Ys], [Y?|Zs?]) :- biased_merge([], Ys?, Zs).
biased_merge([X|Xs], [], [X?|Zs?]) :- biased_merge(Xs?, [], Zs).
\end{verbatim}

The key insight is the \verb|writer(Xs?)| guard in the second clause. This guard succeeds only when the tail of \verb|Xs| is an unbound writer---meaning the producer hasn't yet generated more elements. Only then do we take from \verb|Ys|.

The first clause has no guard and matches whenever \verb|Xs| has an element. Since clauses are tried in order and the first always succeeds when \verb|Xs| is non-empty, elements from \verb|Xs| take priority.

The last three clauses handle termination: when one stream closes (\verb|[]|), drain the other.

\paragraph{Trace Example}

Consider two producers with different speeds:

\begin{verbatim}
fast([1,2,3], done).
slow([a|Tail?], Done?) :- pause, slow_cont(Tail, Done).
slow_cont([b], done).
\end{verbatim}

The fast producer emits all elements immediately. The slow producer emits \verb|a|, pauses, then emits \verb|b|.

\begin{verbatim}
GLP> fast(F, _), slow(S, _), biased_merge(F?, S?, Out).
Out = [1, 2, 3, a, b]
\end{verbatim}

All of \verb|F|'s elements appear first because they're immediately available. The \verb|writer| guard on \verb|S|'s tail fails while \verb|F| has elements. Only after \verb|F| is exhausted does the merge take from \verb|S|.
