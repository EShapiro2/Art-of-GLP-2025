% appendices/app_prolog_to_glp.tex - Prolog to GLP

\chapter{From Prolog to GLP}
\label{app:prolog-to-glp}

This appendix guides Prolog programmers in transitioning to \GLP.

\section{Key Differences}

\begin{center}
\begin{tabular}{lll}
\textbf{Aspect} & \textbf{Prolog} & \textbf{GLP} \\
\hline
Variables & Shared, any can write & Reader/writer pairs \\
Unification & Bidirectional & Writer-only \\
Clause selection & Backtracking & Committed choice \\
Execution & Sequential & Concurrent \\
\end{tabular}
\end{center}

\section{Converting Programs}

\subsection{Variable Usage}

In Prolog:
\begin{verbatim}
swap(pair(X, Y), pair(Y, X)).
\end{verbatim}

In \GLP (requires restructuring):
\begin{verbatim}
swap(pair(X?, Y?), Result) :-
    Result = pair(Y?, X?).
\end{verbatim}

\subsection{Removing Backtracking}

Prolog's generate-and-test patterns must be restructured for committed choice.

\subsection{Adding Concurrency}

Sequential Prolog programs become concurrent \GLP programs through stream-based communication.

\section{Common Patterns}

[Pattern translations to be developed]

\section{Pitfalls}

[Common mistakes to be documented]
