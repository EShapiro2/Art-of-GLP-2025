% appendices/app_history.tex - History

\chapter{Historical Notes}
\label{app:history}

This appendix traces the historical development of ideas leading to \GLP.

\section{Logic Programming Origins}

Logic programming emerged from the intersection of automated theorem proving and programming language design in the early 1970s.

\subsection{Kowalski and Colmerauer}

Robert Kowalski's procedural interpretation of Horn clauses and Alain Colmerauer's Prolog implementation established the field.

\subsection{The Art of Prolog}

Sterling and Shapiro's ``The Art of Prolog'' (1986, 1994) codified logic programming techniques and inspired this book's structure.

\section{Concurrent Logic Programming}

\subsection{Concurrent Prolog}

Shapiro's Concurrent Prolog (1983) introduced committed choice and shared variables for concurrent execution.

\subsection{The FGCS Project}

Japan's Fifth Generation Computer Systems project (1982--1992) developed parallel logic programming languages including GHC (Ueda) and KL1.

\subsection{Flat Concurrent Prolog}

FCP simplified Concurrent Prolog for practical implementation, leading to the Logix system.

\section{The Grassroots Vision}

The grassroots vision emerged from recognizing that:
\begin{enumerate}
\item Centralized platforms create single points of failure and control
\item Cryptographic identity enables trustless authentication
\item Personal devices are powerful enough for distributed computation
\end{enumerate}

\section{From FCP to GLP}

\GLP's key innovation---the SRSW discipline---emerged from recognizing that distributed unification was the fundamental obstacle to grassroots deployment.

\section{Acknowledgments}

[To be written]
