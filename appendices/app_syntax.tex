% appendices/app_syntax.tex - Syntax Reference

\chapter{GLP Syntax Reference}
\label{app:syntax}

This appendix provides a complete formal syntax definition for \GLP.

\section{Lexical Elements}

\subsection{Identifiers}

\begin{itemize}
\item \textbf{Variables}: Start with uppercase letter or underscore
\item \textbf{Constants}: Start with lowercase letter
\item \textbf{Numbers}: Integer or floating-point literals
\end{itemize}

\subsection{Reader/Writer Notation}

\begin{itemize}
\item Writer: plain variable name, e.g., \verb|X|
\item Reader: variable name with \verb|?| suffix, e.g., \verb|X?|
\end{itemize}

\section{Terms}

\begin{verbatim}
term ::= variable
       | constant
       | number
       | functor(term, ..., term)
       | [term | term]            % list
       | []                        % empty list
\end{verbatim}

\section{Atoms and Goals}

\begin{verbatim}
atom ::= predicate(term, ..., term)
       | term = term              % unification
       | term is expression       % arithmetic

goal ::= atom
       | atom, goal               % conjunction
       | (goal)                   % grouping
\end{verbatim}

\section{Clauses}

\begin{verbatim}
clause ::= atom.                  % fact
         | atom :- goal.          % rule
         | atom :- guard | goal.  % guarded rule
\end{verbatim}

\section{Guards}

\begin{verbatim}
guard ::= guard_atom
        | guard_atom, guard       % conjunction

guard_atom ::= ground(term)
             | unknown(term)
             | known(term)
             | term == term       % identity
             | term \== term      % non-identity
             | arithmetic_comparison
\end{verbatim}

\section{Programs}

\begin{verbatim}
program ::= clause
          | clause program
\end{verbatim}

\section{SRSW Requirements}

For a clause to satisfy SRSW:
\begin{enumerate}
\item Each writer variable occurs at most once
\item Each reader variable occurs at most once
\item Variables occur in reader/writer pairs
\end{enumerate}
